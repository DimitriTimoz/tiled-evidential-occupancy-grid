\chapter{Conclusion}

\section*{Conclusion}

In this project, we successfully implemented an evidential mapping approach using the Dempster-Shafer theory of evidence to enhance the performance of mobile robots in dynamic environments. Our experimental results demonstrated the capability of our system to create coherent representations of the environment, detect moving objects, and operate in real-time.

The use of the \textit{SummitXL} robot in a controlled environment allowed us to validate the effectiveness of our method. We were able to generate accurate octomaps that defined the boundaries of the room and detect moving objects. Despite the noise in the IMU data, the program managed to perform reliably, although there is room for improvement in reducing conflicts detected near the boundaries.

Our results suggest several ideas for future work. Methodologically, improvements such as better IMU calibration, sensor fusion, and advanced data filtering could significantly enhance the accuracy and reliability of the system. Implementing optimization techniques such as GPU utilization and memory management can further improve performance. Moreover, exploring dynamic environment adaptation and robust conflict resolution strategies will help for better handling real-world scenarios.

In conclusion, this poject demonstrated the potential of evidential mapping for dynamic object detection in robotics. We developed a real-time system that can be further optimized and improved to enhance the performance of mobile robots in dynamic environments.

